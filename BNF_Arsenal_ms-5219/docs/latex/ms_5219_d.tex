
\documentclass[12pt]{article}
\usepackage{fontspec}
\usepackage[french]{babel}
\setmainfont{Junicode}

% Pacchetto per numerare le righe (deve stare nel preambolo)
\usepackage{lineno}

\begin{document}

% --- PAGINA DEL TITOLO ---
\begin{center}
  {\LARGE BnF. Bibliothèque de l'Arsenal, MS-5219 réserve}\\[2ex]
  {\Large Édition graphématique}\\[2ex]  {\large Lorenzo Paoli}
\end{center}

% Salto pagina per iniziare il testo vero
\newpage

% Inizio numerazione di riga
\modulolinenumbers[5]   % stampa un numero ogni 5 righe
\firstlinenumber{0}     % la prima riga stampata è 1 (poi 5, 10, 15, …)
\linenumbers


    



\newpage
Dela Bibliothẽ. de Charles Adrien Picard
          1742





\subsection*{Prologue du preſent liure par maiſtre robert\\
freſcher bachelier forme en theologie tranſlateur.}


Beroſe ainſi que iozephe nous a laiſſe par eſcript\\
fut natif de la cite de babilone il eſtoit pbr̃e coronique.ͬ\\
et notaire public
          car pour lors loffice des pbrẽs eſtoit\\
de rediger par eſcript le temps que les
          roys regnoyent\\
ensemble leurs geſtes et beaulx faitz comme recite\\
metaſtenes au liure des iugemens des temps. Et a\\
ceſte cauſe
          fiſt ung extrait des fleurs des yſtoires\\
caldaiques en laquelle il a obſerue le temps et anti¬\\
quites depuis le
          deluge qui fut du temps de noe ꝑ\\
eſcript son
          hyſtoire pendent la monarchie dalixãdre\\
le grant qui eſtoit deuant lincarnation enuiron\\
iii.ͨ ans. Et de la creation
          du monde iiii.ͫ uiii.ͨ lix\\
en langue grecque fut treſexpert regent en
            philozophie\\
en luniuerſite dathenes du quel comme dit plineles\\
atheniens auoyent telle extime quilz firent faire a ſõ\\
honneur ung ymaige ayant la langue doree qui fut\\
colloque en vne des
              eſcolles publicque ou il faiſoit\\
les lectures  il a compoſe vng petit volume diuiſe\\
en cinq
            liures. Au premier duquel recite ce que ont\\
eſcript les caldaicques du
          temps depuis adam iuſqẽ\\
au
          deluge. Au ſecond de la genealogie des premiers\\
roys princes et ſeigneurs qui
          premieremẽt regnerent\\
apres le deluge. au iii ͤ de lantiquite de noe et com\\
ment Il fut nomme 
            ianus dont eſt nõe ianuarius\\
le mois de iãuier. au iiii.ͤ des
          antiquites des



\newpage
royaumes de toute la terre en general. Au
            cinquieſ.ͤ\\
dun chaſcun royaume en particulier. A ceſte cauſe\\
ſire
          dautant quil na point obmis a parler du\\
royaume de france comme vous pourres
            ouẏr\\
par lectre de ce pñt liure en pourſuiuant ce que\\
ledit
            beroſe et depuis luy maintes en ont eſcript\\
iuſques a francus dont ſont ditz les
          francoys ien\\
ay fait vng extrait en adiouſtant ce q̃ tholomee\\
 pline dydore Iosephe et autres pluſieurs ont eſcript\\
sur ceſte matiere ce que iay fait affin que plus\\
facillement par aide
          de teſmoins on adiouſte foẏ\\
a ce quilz en ont eſcript. Et ainſi que vr̃e
            treſilluſtre\\
et diuin eſcript eſt touſiours doulz et begnin a\\
receuoir le labeur tant ſoit de petite extime de voz\\
treshumbles ſubgectz et
          moindres ſeruiteurs. Il\\
vous plaira ſire auoir leuure aggreable. Car ie\\
ne lay fait ſinõ affin que vr̃e treſdiuin ẽtendem̃t\\
y preigne
          quelque foy recreation comme bien voꝰ\\
en ſuyuez en lectures de geſtes enciens
          des treſmana¬\\
nimes empereurs roys et princes qui deuant vous\\
ont
          regne. Et neuſſe voulu mettre la main aleuure\\
ſinõ que euidentement par diuers
          acteurs dignes\\
de foẏ iaẏ congneu la choſe qui dedans eſt recitee\\
eſtre veritable. Et croy que apres moyſe qui a eſcript\\
les premiers liures de noſtre ſaincte eſcripture
            nya\\
eu hyſtoriagraphe qui ſoit plus approuche de la\\
verite et
          auſſi que peu de gens qui ont veu la cro\\
nnique auecques lexpoſiteur dicelle a grant peine\\
vouldroyent dire au
          contraire. Et a tant feray fin\\
au prologue pour venir a la proſecution de
          leuure



\newpage






\subsection*{Du temps depuis adam iuſques a noe}


Ante aquarum cladem famoſam qua vniu̾ſus\\
periit orbis
              multa preterierunt ſecula que a noĩs\\
caldeis fideliter fuerunt
              ſeruata. Or dit doncq̃s\\
beroſe que deuant le
          deluge eſtoyent pluſieurs\\
ans paſſes lesquelz ont fidelement obſeruez et\\
redige par eſcript les hyſtoriographes caldaicq̃s\\
Car ainſi que recite
            filo en son breuiaire du\\
temps depuis adam iuſques au deluge ſont cõptez\\
M vi.ͨ lṽi ans. En
          la premiere ſepmaine du\\
premier an comme eſcript moyſeau p̾mier chap¬\\
pitre de geneſe
          fut cree adam qui engendra pluẜ ͬ ᷤ\\
enfans entre les autres au C et xxx.ͤ an de
            ſon\\
eage engendra vng nomme ſeth dont
            deſcendit\\
noe duquel ſont descendus tous les
            hommes\\
de pñt comme ſenſuit. Adam.
            Seth Enos\\
Caynam. Malaleel. Iareth. Enoc. Mathusalẽ\\
Lameth. Noe.



Depuis adam ſont deſcendus les deſſusditz\\
de pere en filz.



Noe en la creation du monde M v.ͨ lviii ans\\
qui fut deuant le deluge iiii ͯ ͯ xviii ans engẽ\\
dra de ſa femme
          nõmee tydea. Sem Iaphet.\\
et cam. et trois filles nõmees
            pendora noella\\
noegla qui ſont huit ꝑſongnages eſquelz fut\\
le
          genre humain conſerue de la lignee de iaphet





\subsection*{ſecond chappitre. Tranſlateur}


Ie laiſſe a parler des enfans leſquelz noe
            ẽgendra\\
apres le deluge ſemblablement auſſi des lignees



\newpage
de ſem et
            cam et deduire la lignee de iaphet ſecond\\
filz de noe car ceſt
          deluy que viendront les p̾miers\\
qui habiterent le pays de france comme ſera
            eſcript\\
cy apres.



Iaphet eut huit enfans qui ſenſuiuent\\
Comerus gallus medus magogus. ſamothes\\
autrement nomme dysTubal et Iubal moſeus\\
TirasIon.



Des huit enfans de iaphet ie veulx ſeullement en\\
ſuiure la lignee du
          iiii.ͤ nomme Samothes duquel\\
beroſe en ſon cinquieſme liure
          parle en telle maniere\\
Samothes en lan Cent quarente apres le
                deluge\\
fut le premier qui fonda et edifia les villes au pays\\
lequel depuis a eſte nomme france cõe ya dit cy apres\\
De ſon temps ne fut
              trouue homme plus ſaige et\\
deluy ceulx qui depuis furent nõmez francoys
                eſtoiẽt\\
appeles ſamothes. Les theologiens et grans philoſophes\\
qui ia eſtoyent deſcendus de ſon grant pere noe et de\\
son pere iaphet et
          de ſes oncles ſem et cam
          et autres\\
que engendra noe apres le deluge
          auoyent led̶ ſa¬\\
mothes a merueilleuse
          reuerance. dyogenes laercius\\
lequel a escript la vie des
            phillozophes qui ia eſtoyẽt\\
deſcendus dit au commancement de ſon liure
            dit\\
que philozophie print ſon commancement des ꝑſes\\
leſquelz
          auoyent tiltres de philozophes appellees\\
mages des babiloniens et aſſiriens ou
          furent appellez\\
caldaicques et des celtes qui ſont ausourduy nõmes\\
francois qui ſe nõmoyent ſamothes leſquelz ſamothes\\
ainſi que teſmoigne
            ariſtote en ſa magicque et s\\
cecio au xxiii.ͤ liure des
            ſuſceſſions furent tres ſauãs



\newpage




en droit diuin et humain Et a ceſte cauſe trſadonnez\\
a Religion. Et
          quelque choſe que veulle dire dyogenes\\
qui recite que philozophie fut premierement trouuee\\
en grece ceſt erreur tout
          manifeſte car ainſi que dit\\
cezar au vi.ͤ liure de ſes cõmentaires toutes les\\
gaulles ont eu les ſciences et ſauoir des. Samothes\\
ainsi que dyent les grecz les auoyent des ſamothes\\
Car deuant le temps de cadmus quilz dyent ĩuẽte.ͬ\\
des lectres elles auoyent eſte en france vſitees ẽſemble\\
auſſi les
          mectres et vers poetiques philozophie theo\\
logie et les loix ainſi appart que
          quelque choſe que\\
veullent dire pluſieurs hiſtoriographes que le p̾mier\\
Roy de france fut filz du ſecond filz de noe qui
            premier\\
diſtribua les ſciences en france et pluſieurs autres\\
regions. Par quoy les grecz ſont faulx et mẽſõgiers\\
qui se veullent
          atribuer tel honne.ͬ Car longue\\
eſpace de temps deuant que aucune ſcience ne
            lictera\\
ture floriſt en grece les francois allemans espaignoulz\\
et
          autres nations les auoyent comme familieres\\
Ceſtuy ſamothes premier roy de france euſt telle\\
poſterite et
          lignee.



Samothes engendra galathea. galathes magus\\
ſarron. Nannes dryudes bardus longo bardus ii ͤ\\
de ce nom.
          Seltes





\subsection*{Du ii ͤ Roy ͤdes Seltes que nous appellons\protect\\ francoys nomme
            magus chappitre iii ͤ}


Magus premier filz de ſamothes regna apres ſõ pere.





\subsection*{Tranſlateur}


\newpage
Il eſt a noter que ce nom ycy magus eſt vng tm̃e\\
ou vocable. Stithien qui vault
          autant a dire\\
comme edificateur ou batiſſeur de maiſons\\
car les
          francois vſerent adoncques de ceste lan¬\\
gue bien vray eſt que les perſes disent
          ce nom\\
magus ſignifier philozophe mais en ce temps\\
la langue des
          perſes neſtoit encore en vſaige\\
Beroſe dit
          que ceſtuy cy fut le premier qui edifia\\
pluſieurs villes et actes . Car comme
            teſmoigne\\
tholomee fut par luy edifiee
          en acquitaine vne\\
cite de luy nommee nomomagus en la gaulle\\
lyonnoiſe neomagus rothomagus et nyoma¬\\
qus en la france belgicque vng autre
            rothoma\\
gus qui eſt la cite principale de normendie\\
nommee Rouen
          et autre nõmee neomagus\\
Berbethomagus en la gaulle de nerbonne\\
edifia deux cites. vindomagus et neomagus\\
Et uilles ceſar en ſes commentaires a nõme\\
en france
          iuliomagus et teſaromagus . Car\\
toutes cites de ce nom magus eſtoyent
            appellees\\
magez ie me de porte de dire quelles cites ce ſont\\
dautant que aucunes depuis le temps de tholo¬\\
mee ont eſtes deſtruictes et les autres ont chãge\\
leur nom.
          Touteſfois il eſt euident q̃ rothoma¬\\
gus eſt rouen en normendie.





\subsection*{De ſarron ii.ͤ filz de ſamothes troiſieſme\protect\\ Roy des celtes ou francoys
          chiiii ͤ}


Sarron regna apres ſon frẽ magus qui\\
fut la pñt qui par auſter aux hoẽs les
          mauuaiz



\newpage




exteraces fiſt les eſcolles publicques que nous\\
diſons vniuersites.
          Et de luy furent les ſamothes\\
au Iourduy francoys appelles. Saronides aĩſi\\
que dit diodore en son vi.ͤ liure il ya dist il en\\
la
          france celticque pluſieurs theologiens et phi¬\\
loz ophes qui ſappellent
          zaronides qui ſont gens\\
de grant eſtime entre les hoẽs et ont telle couſtuẽ\\
que iamais ilz ne font ſacriffice que quelque\\
grant philozophe ne ſoit
          leur principal directe.ͬ\\
car ilz ont eſtime que les ſaiges ont participatiõ\\
auecques la nature diuine et que par eulx ſe\\
doiuent faire les
          ſacrifices et que len ne doit riẽs\\
demander aux dieux ſinon par linterceſſiõ
            diceulx\\
et que en paix et guerre on doit vſer de le.ͬ ꝯseil\\
Bien
          est vray que deuant ceſtuy ſarron les ſa\\
mothes eſtoyent theologiens et philozophes\\
mais ilz nauoyent encores aucunes
            eſcolles\\
publicques.





\subsection*{Du iiii.ͤ roy de celtes ou francois nomme\protect\\ dryus v.ͤ chappitre.}


Dryus iiii.ͤ Roy de france fut homme de\\
treſgrant et merueilleux ſauoir. Ceſar en
            ſon\\
ſix.ͤ des commentaires et
            diodore en ſon ſix.ͤ\\
liure des celtes
          diſent ilz qui ſont les francoys\\
vsent de diminutions dangures et sacrifices\\
par leſquelles prediſoyent les choſes futures de\\
ceulx Icy parle lucan et
            plineen la fin du\\
xvi.ͤ liure de
            ſuetone de ceulx cy parle
            lucan\\
et plineen la fin en la uie de claude cezar Ce



\newpage
ſont ceulx de dreux et des enuirons en
          normẽdie





\subsection*{De bardus v.ͤ Roy des celtes chappitre
          vi ͤ}


Bardus fut treſexcellant en Inuention de
            ꝯposer\\
vers et chancons de muſicque ainſi que dit beroze\\
de ceſtuy vint vne ſecte nõmee la ſecte bardre que\\
comme alaiſſe par eſcript diodore
          en ſon vi.ͤ liure\\
il ya
          diſt il en gaulle celtiques poetes inuenteurs\\
de toute doulceur de muſique quon
          appelle bardes\\
et chantent auecques les orgres ne plus ne mains\\
que
          les autres nations auecques la herpe diſans\\
louenges des vngs et vitupere des
          autres et ſont de\\
ſi grande eſtime en guerre quilz nont pas ſi toſt\\
veu leurs ennemis quilz ſont vaincus et pource\\
ſauue lonneur des grecz et ne
          leur deſplaiſe car ilz\\
ont eu les ſciences des francoys quilz ont reputes\\
barbares qui ont premier trouue les ars et ſciences\\
que nont les
          grecz





\subsection*{De longho vi.ͤ Roy des celtes}


Longho fut le premier duquel ſont nommes\\
ceulx de langres lesquelz de ce nom eſtoyent ditz\\
longones en terme
          latin et maintenant lingones\\
comme recite ptholomee et autres triographes\\
Ce nom ycy longho vault autant adire cõe
            prince\\
conſeruant et de grant prudence a garder ce quil\\
auoit
          conquis





\subsection*{De bardus vii ͤ roy des celtes}










\newpage




Bardus le ieune et ſecond de ce nom
          Regna apres\\
longho de luy neſt trouue choſe
          digne de memoire





\subsection*{De lucus viii.ͤ Roy de
          celtes}


Lucus eſt celluy dont ſont nommes au pres
            de\\
paris luſarche comme le teſmoigne ptholomee en\\
ſa coſmagraphie.





\subsection*{De celtes ix.ͤ Roy des
          celtes ou francois}


Celte est celuy dont eſt nomme la gaulle
            Celtique\\
qui fut la premiere terre hĩtee en france ce nom\\
de celtes
          leur a dure iusques a francois xxii.ͤ Roy\\
dont fut le pays nomme france et les
            hommes\\
francois qui en portent le nom par excellance es\\
gaulles
          ſubgete au roy de france en ſorte que toutes\\
les terres de ſon Royaume comme
          acquitaine et la\\
gaulle belgicque et autres terres voismes ſont ap¬\\
pellees france et les hĩtans dicelle francois. Et le roy\\
de france prent ſon
          nom de celle Region dautant\\
que apres le deluge fut la premiere habitee Et
            quil\\
ſoit ainsi lexperience nous le monſtre a loeil et a\\
vnchaſcun
          viuant de ce temps qui vouldroit ſau̾\\
letymologie de ce nom celtes liſe
          viterbience cõmẽtate ͬ\\
de beroſe surce
          paſſaige en ſon cinq.ͤ liure ſur le tex¬\\
te apud celtas celte aquo ⁊c̃.





\subsection*{De galathes x.ͤ Roy de celtes}


Galathes fut filz de hercules degipte aniſi que
          dit



\newpage
beroſe
          car apres que hercules eut ſurmonte les
            tyrãs\\
en eſpaigne il voulut aller en ytallye et print ſon\\
chemin
          par les gaulles ou en paſſant par les gaulles\\
vint veoir le roy deſſuſd̶
            Celtes et dune ſienne\\
fille nõmee galathea leur engendra ceſtuy galathes\\
dont vint que les celtes furent premierement nõmes\\
galles ou gallois. Et que ainſi ſoit apres beroſe\\
diodore en
          son six.ͤ nous a laiſe par eſcript ce
            qui\\
enſuit. Le temps paſſe y euſt ung treſnoble\\
prince qui regnoit
          aux parties celtiques ou fran¬\\
coiſes lequel auoit vne fille exedente la
            cõmune\\
nature en grandeur et beaulte et elegance de corps\\
treſexcellante en vertu force et couraige. A ceſte\\
cauſe depriſoit tous les
          hommes qui la vouloyent\\
auoir pour femme et neſtimoit homme ſouffiſãt\\
ne digne de lauoir en mariage. Le temps pendãt\\
hercules a ſon retour deſpaigne ou long tep̃s
            auoit\\
eſte et ediffie la cite nõmee alexia apres la victoire\\
obtenue contre berion. Ceſte fille du roy
            galathea\\
voyant par admonition les
          grandes vertus de\\
hercules la
          beaulte et grandeur de ſon corps fut\\
de luy amoureuſe et de conſentement de
          pere et de\\
mere euſt la fruiction de ſon amour . Et de luy\\
conteupt
          vng filz nõme galathes lequel entre toꝰ\\
ceulx de ſon temps en vertuz et puiſſance de corps\\
eut le nom le bruit et la
          renommee par ſes gran¬\\
des vertuz et puiſſance luy venu en eage dhõme\\
et ſucceſſeur du royaume apres celtes pere de
            gala¬\\
thea sa mere par ſes geſtes et
          vertuz conquiſt pluẜ ͬ ᷤ\\
terres et ſeigneuries et augmenta ſon royaume\\
Et quant il ſe uit en gloire et puiſſance et eſtre



\newpage




bien obey de ſes ſubgetz il voulut et decreta que de\\
ſon nom feuſt
          nomme tout ſon peuple Et furẽt\\
ditz galathes et toute la region gallicanne
            euſt\\
ce nom et tout le pays. Et ſont les parolles de\\
diodore dont il appert que du temps de hercules\\
ceſte region galicanne euſt ce nom
          qui depuis\\
par ſuſtraction de leurs a eſte dicte gallia en memoi¬\\
re
          de ce Roy galathes filz de hercules. Auecques ce\\
que nous auons deſſus
          allegue comme ſolinus
            qui\\
eſt acteur comoſgraphe de grant eſtime lequel dit\\
que les
          galathes du pays daſie leſquelz eſcript ſainct\\
paolad galathas ont eu leur origine des
          galles q̃\\
nous diſons francois. Hercules en ce meſmes teps̃\\
edifia en bourgongne vne cite nomme
          en langue\\
latine au Iour duy alſeta qui neſt pas loing dauthũ\\
Oultre plus recite diodore que les
          galathes a com¬\\
paignes de pluſieurs autres leurs vvoiſins vindret̃\\
en grece occuper le pays daſye ou fonderent et dõne¬\\
rent premierement le
          nom aux gallathes. ausquelz\\
ſainct
            paol a dirige et eſcript plusieurs epitres. Et\\
furent appellez
          ainſi que iosephe et ſolinusraconte\\
et tesmoignent ceulx qui firent la
          conqueſte des\\
grecz gallatoys sunt gallogreci. A ceste cauſe veult\\
inſerer que dautant que luſaige des lettres eſtoit\\
entre les galloẏs comme
          eſt deſſuſd̶ La raison eſt\\
euidente a ce que dyent ceulx de meonie qͥlz
            auoyẽt\\
premierement les lectres des francois. et que les\\
grecz
          les eurent deulx et non pas de cadmus quilz\\
dient auoir eſte celluẏ qui premier les apporta\\
des phenierons aux grecz
          comme ilz ont faulcemt̃\\
de leur couſtume meſcongneu. Ad ce propos auſſi



\newpage
ſert ce que manethon racompte que ala fin du regne\\
de ilus Roy de troye galathes ſecond de ce nom ſur\\
monta les ſamothes et tanydicques et fonda pluẜ ͬS\\
villes es gallathes par
          quoy nous eſt euidente pro¬\\
bation que la puiſſance des galles perſeuera et
            con¬\\
tinua longuement en aſẏe. Et pour ceſte cause appa¬\\
roiſt que
          les ars ſciences et lettres ont eu origine\\
des francois ce que les grecz ſe ſont
            faulcement\\
atribues ce loz ce que iamais ne fut trouue en librai¬\\
rie ne ſuffisant acteur auſſi que leur reprouche ioze¬\\
phus en ſon liure contre la grammaire appion.
            Et\\
pour faire fin en ceſte matiere diodore dificilledesicille dit po ͬ\\
la verite que les rommains appellent les gallathes\\
ceulx que les autres nomment les celtes qui ſont les\\
francors. ainsi appart
          comie les francois quilz\\
furent premierement nõmez ſamothes de leur p̾mier\\
Roy muerent leur nom par pluſieurs foys. Car de\\
ſamothes apres furent
          appelles celtes de leur Roy\\
celthe. Apres
          furent appelles ou nõmes galathes\\
et leur regne augmente de gaulle belgicque
            par\\
vng nomme belgius.
          Et comme nous verrons\\
furent appelles galles belgicques les rommains le ͬ\\
ont touſiours donne ce nõ galli qui ſont galles\\
Et eſt celuy qui
          entre les hyſtoriens eſt plus vſite\\
finablement comme nous verrons a la fin de
            ce\\
liure furent appelles francois qui eſt le terme plꝰ\\
cõmun de
          pardeca en nr̃e langue vulgaire. Par\\
ainſi il appart que galathes filz de hercules donna\\
ce nom gaulles au pays et aux hommes de france\\
Le quel galathes du temps de altadesxii.ͤ Roẏ\\
de babilloyne.



\newpage






\subsection*{De harbon xi.ͤ Roẏ de
          celtes}


Harbon en muant .h. en .n. eſt narbon et de
            luẏ\\
fuc dit le paẏs de narbonne il eſtoit filz de galathes\\
de ceſte galle de narbonne par lepline en ſon iii ͤ\\
liure de liſtoire naturelle. La prouince dit que de\\
narbonne eſt vne partie des gaulles et eſt au de\\
ſoubz de lamer
          deſpaigne et ſe nommoit galia\\
broceata diuiſee de itallie par la riuiere
            nommee\\
en latin varus il ya homme deſperit fertilite de t̾re\\
multitude de richeſſes par quoy ne ſcay prouince\\
digne deſtre preferee au
          pays dytalie ſinõ ceſte cy





\subsection*{De lugdus xii.ͤ Roy de
          celtes.}


Lugdus eſt celluy dont la prouince prent
            ſon\\
nom comme recite beroſe et ainſi q̃ le
          nõ laccuſe\\
dit maiſtre iehan de viterbe
          cõmentate.ͬ dud̶ beroſe\\
ceſt celuy dont eſt
          nõme lugdiniũ qui eſt lyon en\\
la gaulle lugdunẽse ou lionnoiſe on appelle de
            ceſtuy\\
lugdus les hommes ludouici et nõ
          obstãt g. demeure\\
ludouici q̃ nous diſons en francoys loys





\subsection*{De belgius xiii.ͤ Roy de celtes.}


Belgius comme dit beroſe est celuy dont eſt dicte\\
la gaulle belgieque qui commance
          depuẏs le paẏs\\
ou paſſe la riuiere de cene et marne du coſte de sep\\
tentrion iuſques au fleuue de rouen et vne partie\\
vers orient du coſte
          doccident finiſt a la mer octeane\\
qui eſt la mer de bretaigne.





\newpage
\subsection*{Du Roy allobrox
          xiiii.ͤ Roy de celtes ou francois}


Allobrox ainſi que dit viterbroucequaſi alle¬\\
droys qui eſt adire meridionnale mixtion car\\
drox
              ou daros eſt le uent de mydy que les hebrai¬\\
ques dyent daros et hallo eſt
              adire mixtion a\\
ce conſentent ſainct iheroſme et
          les talmudiſtes\\
et de ces deux dictions vient allodarox ou allo¬\\
drox
          qui depuis a eſte dit allobrox qui eſt ĩter¬\\
prete cõmixtion ſont ditz les allobroges qui ont\\
pluẜ ͬ ᷤcites
          treſanciennes comme dyent ptholome\\
et
          autres coſmographes entre leſquelz est viẽne\\
geneue et auignon meridyonale il
          neſt adou¬\\
ter q̃ de ce roy allobrox.





\subsection*{Translateur}


Icy finiſt beroſe qui a eſcript ceulx qͥ
            regnerent\\
en diu̾ſes parties de la terre apres le deluge iuſqs̃\\
a
          ſon temps



Maneton cõmance ou beroſe afiny et ꝯtient\\
iuſques a ſon temps duquel regna francus
            dõt\\
ſont ditz les francois cõe len verra cy apres.





\subsection*{Du xv.ͤ Roy des celtes ou francois nõe Romus}


Romus eſt celluy dit manethon dont ſont ditz\\
romandi ptholomee dit q̃ en sa gaulle belgicq̃\\
ya les
          romandiſſes ſur le fleuue de roſne au de\\
ſoubz de vienne romonũ et roma qui
          depuis a\\
eſte nõmee ꝑ interpretaᵒ.ᷠ ualence.







\newpage






\subsection*{Du xvi.ͤ Roy de celtes ou francois nõe paris}


Paris fut celluy dont eſt dicte la cite de
            paris\\
qui par auant auoit nom lutocia qui de nr̃ẽ\\
temps eſt
          treſſouueraine vniu̾se du mõde en\\
la quelle fut docteur regent ſainct thomas daquĩ\\
ptholomee en ſa geographie et aut̾s pluẜ.ͬ ᷤ





\subsection*{du xvii.ͤ Roy de celtes nõe le
            mãnus.}


Le mãnus fut celuy dont sont ditz le
            main\\
dont eſt le lac que nous appellõs Luſanne





\subsection*{du xviii.ͤ Roy de frãce nomme olbius}


Olbius dit manetõ Regna apres le Roy mãmus\\
et de luy neſt trouue choſe digne de memoire.





\subsection*{du xix.ͤ Roy de celtes ou frãcois nõe galathas\\
ſecond de ce nom.}


Gallatas deux.ͤ de ce nom fut celluy
          qui ſurmõ\\
ta les ſamothes en aſye et edifia pluſieurs villes\\
en
          france en lhonneur de ſon pere olbius et es\\
autres regions par luẏ ſubjuguees comme en\\
ſamothe aſie et Cilicie ainſi que
            ptholome et\\
autres geographes nõment
          pluſieurs cites\\
olbiances de ce nom olbius.
          ⁊c̃.





\end{document}
    

\documentclass[12pt]{article}
\usepackage{fontspec}
\usepackage[french]{babel}
\setmainfont{Junicode}

% Pacchetto per numerare le righe (deve stare nel preambolo)
\usepackage{lineno}

\begin{document}

% --- PAGINA DEL TITOLO ---
\begin{center}
  {\LARGE BnF. Bibliothèque de l'Arsenal, MS-5219 réserve}\\[2ex]
  {\Large Édition semi-diplomatique}\\[2ex]  {\large Lorenzo Paoli}
\end{center}

% Salto pagina per iniziare il testo vero
\newpage

% Inizio numerazione di riga
\modulolinenumbers[5]   % stampa un numero ogni 5 righe
\firstlinenumber{0}     % la prima riga stampata è 1 (poi 5, 10, 15, …)
\linenumbers


    



\newpage
De la Bibliotheque. de Charles Adrien Picard
          1742





\subsection*{Prologue du present livre par maistre Robert\\
Frescher bachelier forme en theologie translateur.}


Berose ainsi que Jozephe nous a laisse par escript\\
fut natif de la cite de Babilone il estoit prestre coroniqueur\\
et notaire
          public car pour lors l'office des prestres estoit\\
de rediger par escript le
          temps que les roys regnoyent\\
ensemble leurs gestes et beaulx faitz comme
            recite\\
Metastenes au livre des jugemens des temps. Et a\\
ceste cause
          fist ung extrait des fleurs des ystoires\\
caldaiques en laquelle il a observe le temps et anti¬\\
quites depuis le
          deluge qui fut du temps de Noe par\\
escript son
          hystoire pendent la monarchie d'Alixandre\\
le grant qui estoit devant l'incarnation environ\\
300 ans. Et de la creation
          du monde 4859\\
 en langue grecque fut tres expert regent en philozophie\\
en l'universite d'Athenes du quel comme dit Plineles\\
Atheniens avoyent telle extime qu'ilz firent faire a son\\
honneur ung ymaige ayant la langue doree qui fut\\
colloque en une des
              escolles publicque ou il faisoit\\
les lectures  il a compose ung petit volume divise\\
en cinq
            livres. Au premier duquel recite ce que ont\\
escript les caldaicques du
          temps depuis Adam jusque\\
au
          deluge. Au second de la genealogie des premiers\\
roys, princes et seigneurs qui
          premierement regnerent\\
apres le deluge. au III ͤ de l'antiquite de Noe et com\\
ment Il fut nomme 
            Janus dont est nomme ianuarius\\
le mois de janvier. au IV ͤ des
          antiquites des



\newpage
royaumes de toute la terre en general. Au
            cinquiesme\\
d'un chascun royaume en particulier. A ceste cause\\
sire
          d'autant qu'il n'a point obmis a parler du\\
royaume de France comme vous pourres
            ouyr\\
par lectre de ce present livre en poursuivant ce que\\
ledit
            Berose et depuis luy maintes en ont escript\\
jusques a Francus dont sont ditz les
          Francoys j'en\\
ay fait ung extrait en adjoustant ce que Tholomee\\
 Pline Dydore Josephe et autres plusieurs ont escript\\
sur ceste matiere ce que j'ay fait affin que plus\\
facillement par
          aide de tesmoins on adjouste foy\\
a ce qu'ilz en ont escript. Et ainsi que
          vostre tresillustre\\
et divin escript est tousjours doulz et begnin a\\
receuoir le labeur tant soit de petite extime de voz\\
treshumbles subgectz et
          moindres serviteurs. Il\\
vous plaira sire avoir l'euvre aggreable. Car je\\
ne l'ay fait sinon affin que vostre tresdivin entendement\\
y preigne
          quelque foy recreation comme bien vous\\
en suyvez en lectures de gestes enciens
          des tresmana¬\\
nimes empereurs roys et princes qui devant vous\\
ont
          regne. Et n'eusse voulu mettre la main a l'euvre\\
sinon que evidentement par
          divers acteurs dignes\\
de foy j'ay congneu la chose qui dedans est recitee\\
estre veritable. Et croy que apres Moyse qui a
            escript\\
les premiers livres de nostre saincte
            escripture n'y a\\
eu hystoriagraphe qui soit plus approuche de la\\
verite et aussi que peu de gens qui ont veu la cro\\
nnique avecques lexpositeur
            d'icelle a grant peine\\
vouldroyent dire au contraire. Et a tant feray
            fin\\
au prologue pour venir a la prosecution de l'euvre



\newpage






\subsection*{Du temps depuis Adam jusques a Noe}


Ante aquarum cladem famosam qua universus\\
periit orbis
              multa preterierunt secula que a nobis\\
caldeis fideliter fuerunt
              servata. Or dit doncques\\
Berose que devant le
          deluge estoyent plusieurs\\
ans passes lesquelz ont fidelement observez et\\
redige par escript les hystoriographes caldaicques\\
Car ainsi que recite
            Filo en son breviaire du\\
temps depuis Adam jusques au deluge sont comptez\\
1656 ans. En la
          premiere sepmaine du\\
premier an comme escript Moyse au premier chap¬\\
pitre de genese
          fut cree adam qui engendra plusieurs\\
enfans entre les autres au CXXX ͤ an de
            son\\
eage engendra ung nomme Seth dont
            descendit\\
Noe duquel sont descendus tous les
            hommes\\
de present comme s'ensuit. Adam.
            Seth Enos\\
Caynam. Malaleel. Iareth. Enoc. Mathusalem\\
Lameth. Noe.



Depuis adam sont descendus les dessusditz\\
de pere en filz.



Noe en la creation du monde 1558 ans\\
qui fut devant le deluge 88 ans engen\\
dra de sa femme nommee Tydea. Sem Japhet.\\
et Cam. et trois filles nommees
            Pendora,  Noella,\\
 Noegla, qui sont huit persongnages esquelz fut\\
le genre humain
          conserve de la lignee de Japhet





\subsection*{Second chappitre. Translateur}


Je laisse a parler des enfans lesquelz Noe
            engendra\\
apres le deluge semblablement aussi des lignees



\newpage
de sem et
            Cam et deduire la lignee de Japhet second\\
filz de Noe car
          c'est de luy que viendront les premiers\\
qui habiterent le pays de France comme
          sera escript\\
cy apres.



Japhet eut huit enfans qui s'ensuivent\\
Comerus gallus ,  Medus,  Magogus.
            Samothes\\
autrement nomme
            Dys, 
            Tubal et Iubal moseus,\\
Tiras Ion.



Des huit enfans de Japhet je veulx seullement en\\
suivre la lignee du
          IV ͤ nomme Samothes duquel\\
Berose en son cinquiesme livre
          parle en telle maniere\\
Samothes en lan Cent quarente apres le
                deluge\\
fut le premier qui fonda et edifia les villes au pays\\
lequel depuis a este nomme France comme ya dit cy apres.\\
De son temps ne
              fut trouve homme plus saige et\\
de luy ceulx qui depuis furent nommez
              francoys estoient\\
appeles Samothes. Les theologiens et grans
              philosophes\\
qui ia estoyent descendus de son grant pere Noe et de\\
son pere Japhet et
          de ses oncles Sem et Cam 
          et autres\\
que engendra Noe apres le deluge
          auoyent ledit Sa¬\\
mothes a merueilleuse
          reuerance. Dyogenes Laercius\\
lequel a escript la vie des
            phillozophes qui ia estoyent\\
descendus dit au commancement de son livre
            dit\\
que philozophie print son commancement des Perses\\
lesquelz
          auoyent tiltres de philozophes appellees\\
mages des Babiloniens et Assiriens ou
          furent appellez\\
caldaicques et des Celtes qui sont aujourduy nommes\\
Francois qui se nommoyent Samothes lesquelz Samothes\\
ainsi que tesmoigne
            Aristote en sa Magicque et S\\
cecio au XXIII ͤ livre des
            suscessions furent tres savans



\newpage




en droit diuin et humain. Et a ceste cause tresadonnez\\
a Religion.
          Et quelque chose que veulle dire Dyogenes\\
qui recite que philozophie fut premierement trovuee\\
en Grece c'est
          erreur tout manifeste, car ainsi que dit\\
Cezar
          au VI ͤ livre de ses Commentaires toutes
            les\\
Gaulles ont eu les sciences et sauoir des Samothes\\
ainsi que dyent les grecz les avoyent des Samothes\\
. Car devant le temps de Cadmus qu'ilz dyent
            inventeur\\
des lectres elles auoyent este en france usitees ensemble\\
aussi les mectres et vers poetiques philozophie theo\\
logie et les loix ainsi
          appart que quelque chose que\\
veullent dire plusieurs historiographes que le
            premier\\
Roy de France fut filz du second filz de Noe qui premier\\
distribua les sciences en France et plusieurs
            autres\\
regions. Par quoy les Grecz sont faulx et mensongiers\\
qui
          se veullent atribuer tel honneeur. Car longue\\
espace de temps devant que aucune
          science ne lictera\\
ture florist en Grece les Francois, Allemans, Espaignoulz\\
et autres nations les auoyent comme familieres.\\
Cestuy Samothes premier roy de france eust telle\\
posterite et lignee.



Samothes engendra Galathea. Galathes, Magus,\\
Sarron. Nannes,
           Dryudes,
           Bardus,
           Longo,
           Bardus II\\
de ce nom.
          Celtes





\subsection*{Du II ͤ Roy des Celtes que nous appellons\protect\\ Francoys nomme
            Magus chappitre III ͤ}


Magus premier filz de Samothes regna apres son pere.





\subsection*{Translateur}


\newpage
Il est a noter que ce nom ycy
            Magus est ung terme\\
ou vocable. Scithien qui
          vault autant a dire\\
comme edificateur ou batisseur de maisons\\
car les
          Francois userent adoncques de ceste lan¬\\
gue bien vray est que les Perses disent
          ce nom\\
magus signifier philozophe mais en ce temps\\
la langue des
          Perses n'estoit encore en usaige.\\
Berose dit
          que cestuy cy fut le premier qui edifia\\
plusieurs villes et actes. Car comme
            tesmoigne\\
Tholomee fut par luy edifiee
          en Acquitaine une\\
cite de luy nominee Nomomagus, en la Gaulle\\
lyonnoise Neomagus, Rothomagus et Nyoma¬\\
gus en la France belgicque, ung
          autre Rothoma\\
gus qui est la cite principale de Normendie\\
nommee
          Rouen et autre nommee Neomagus,\\
Berbethomagus, en la Gaulle de Nerbonne\\
edifia deux cites, Vindomagus et Neomagus\\
Et  Jules Cesar  en ses Commentaires a nõme\\
en france Juliomagus et Tesaromagus . Car\\
toutes cites de ce nom Magus estoyent appellees\\
magez, je me de porte
          de dire quelles cites ce sont\\
d'autant que aucunes depuis le temps de Tholo¬\\
mee ont estes destruictes et les autres ont
            change\\
leur nom. Toutesfois il est euident que Rothoma¬\\
gus est
          Rouen en Normendie.





\subsection*{De Sarron II ͤ filz de Samothes troisiesme\protect\\  Roy des Celtes ou Francoys
          chapitre IV ͤ}


Sarron regna apres son frere Magus qui\\
fut le premier qui par auster aux hommes les
          mauvaiz



\newpage




exteraces fist les escolles publicques que nous\\
disons universites.
          Et de luy furent les Samothes\\
aujourduy Francoys appelles Saronides ainsi\\
que dit Diodore en son VI ͤ livre il y a dist il en\\
la France celticque plusieurs theologiens et phi¬\\
lozophes qui s'appellent
          Saronides qui sont gens\\
de grant estime entre les hommes et ont telle
            coustume\\
que jamais ilz ne font sacriffice que quelque\\
grant
          philozophe ne soit leur principal directeur\\
car ilz ont estime que les saiges
          ont participation\\
avecques la nature divine et que par eulx se\\

          doivent faire les sacrifices et que l'en ne doit riens\\
demander aux dieux sinon
          par l'intercession d'iceulx\\
et que en paix et guerre on doit user de lleur
            conseil.\\
Bien est vray que devant cestuy Sarron les Sa\\
mothes estoyent theologiens et philozophes\\
mais ilz n'avoyent encores aucunes escolles\\
publicques.





\subsection*{Du IV ͤ roy de Celtes ou Francois nomme\protect\\ Dryus V ͤ chappitre.}


Dryus IV ͤ Roy de France fut homme de\\
tresgrant et merveilleux savoir. Cesar en
            son\\
sixieme des Commentaires et
            Diodore en son sixieme\\
livre des Celtes
          disent ilz qui sont les Francoys\\
usent de divinations d'augures et sacrifices\\
par lesquelles predisoyent les choses futures de\\
ceulx Icy parle Lucan et
            Plineen la fin du\\
XVI ͤ livre de
            Suetone de ceulx cy parle
            Lucan\\
et Plineen la fin en la vie de Claude Cezar. Ce



\newpage
sont ceulx de Dreux et des environs en
          Normendie.





\subsection*{De Bardus V ͤ Roy des Celtes chappitre
          VI ͤ}


Bardus fut tresexcellant en invention de
            composer\\
vers et chansons de musicque ainsi que dit Beroze\\
de cestuy vint une secte nommee la secte bardre que\\
comme a laisse par escript Diodore
          en son VI ͤ livre\\
il y a
          dist il en Gaulle celtiques poetes inventeurs\\
de toute doulceur de musique qu'on
          appelle Bardes\\
et chantent auecques les orgres ne plus ne mains\\
que
          les autres nations avecques la herpe disans\\
louenges des ungs et vitupere des
          autres et sont de\\
si grande estime en guerre qu'ilz n'ont pas si tost\\
veu leurs ennemis qu'ilz sont vaincus et pource\\
sauve l'onneur des Grecz et
          ne leur desplaise car ilz\\
ont eu les sciences des Francoys qu'ilz ont
            reputes\\
barbares qui ont premier trouve les ars et sciences\\
que
          nont les Grecz





\subsection*{De Longho VI ͤ Roy des Celtes}


Longho fut le premier duquel sont nommes\\
ceulx de Langres lesquelz de ce nom estoyent ditz\\
Longones en terme
          latin et maintenant Lingones\\
comme recite Ptholomee et autres triographes.\\
Ce nom ycy Longho vault autant a dire comme
            prince\\
conservant et de grant prudence a garder ce qu'il\\
avoit
          conquis.





\subsection*{De Bardus VII ͤ roy des Celtes}










\newpage




Bardus le jeune et second de ce nom
          regna apres\\
Longho de luy n'est trouve chose
          digne de memoire.





\subsection*{De Lucus VIII ͤ Roy de
          Celtes}


Lucus est celluy dont sont nommes au pres
            de\\
Paris lusarche comme le tesmoigne Ptholomee en\\
sa Cosmagraphie.





\subsection*{De Celtes IX ͤ Roy des
          Celtes ou Francois}


Celte est celuy dont est nomme la gaulle
            Celtique\\
qui fut la premiere terre habitee en France ce nom\\
de
          Celtes leur a dure jusques a Francois XXII ͤ Roy\\
dont fut le pays nomme France
          et les hommes\\
Francois qui en portent le nom par excellance es\\
Gaulles
          subgete au roy de France en sorte que toutes\\
les terres de son Royaume comme
          Acquitaine et la\\
Gaulle belgicque et autres terres voisines sont ap¬\\
pellees France et les habitans d'icelle Francois. Et le roy\\
de France prent
          son nom de celle Region d'autant\\
que apres le deluge fut la premiere habitee.
          Et qu'il\\
soit ainsi l'experience nous le monstre a l'oeil et a\\
un
          chascun vivant de ce temps qui vouldroit savoir\\
l'etymologie de ce nom Celtes
          lise Viterbience commentateur\\
de Berose sur
          ce passaige en son cinquieme livre sur le tex¬\\
te apud celtas celte a quo etc.





\subsection*{De Galathes X ͤ Roy de Celtes}


Galathes fut filz de Hercules d'Egipte ainsi que
          dit



\newpage
Berose
          car apres que Hercules eut surmonte les
            tyrans\\
en Espaigne il voulut aller en Ytallye et print son\\
chemin
          par les Gaulles ou en passant par les Gaulles\\
vint veoir le roy dessusdit
            Celtes et d'une sienne\\
fille nommee
            Galathea leur engendra cestuy Galathes\\
dont vint que les Celtes furent
          premierement nommes\\
Galles ou Gallois. Et que ainsi soit apres Berose\\
Diodore en son sixieme
          nous a laise par escript ce qui\\
ensuit. Le temps passe y eust ung tresnoble\\
prince qui regnoit aux parties celtiques ou fran¬\\
coises lequel avoit
          une fille exedente la commune\\
nature en grandeur et beaulte et elegance de
            corps\\
tresexcellante en vertu, force et couraige. A ceste\\
cause
          deprisoit tous les hommes qui la vouloyent\\
avoir pour femme et n'estimoit homme
            souffisent\\
ne digne de l'auoir en mariage. Le temps pendant\\
Hercules a son retour d'Espaigne ou long
          temps avoit\\
este et ediffie la cite nommee Alexia apres la victoire\\
obtenue contre Gerion. Ceste fille du roy
            Galathea\\
voyant par admonition les
          grandes vertus de\\
Hercules la
          beaulte et grandeur de son corps fut\\
de luy amoureuse et de consentement de
          pere et de\\
mere eust la fruiction de son amour . Et de luy\\
conceupt
          ung filz nomme Galathes lequel entre tous\\
ceulx de son temps en vertuz et puissance de corps\\
eut le nom, le bruit et la
          renommee par ses gran¬\\
des vertuz et puissance luy venu en eage d'homme\\
et successeur du royaume apres Celtes pere de
            Gala¬\\
thea sa mere, par ses gestes et
          vertuz conquist plusieurs\\
terres et seigneuries et augmenta son royaume.\\
Et quant il se vit en gloire et puissance et estre



\newpage




bien obey de ses subgetz il voulut et decreta que de\\
son nom feust
          nomme tout son peuple. Et furent\\
ditz Galathes et toute la region gallicanne
            eust\\
ce nom et tout le pays. Et sont les parolles de\\
Diodore dont il appert que du temps de Hercules\\
ceste region galicanne eust ce nom
          qui depuis\\
par sustraction de leurs a este dicte Gallia en memoi¬\\
re
          de ce Roy Galathes filz de Hercules. Auecques ce\\
que nous auons dessus
          allegue comme Solinus
            qui\\
est acteur comosgraphe de grant estime lequel dit\\
que les
          Galathes du pays d'Asie lesquelz escript sainct\\
Paol ad Galathas ont eu leur origine des
          Galles que\\
nous disons Francois. Hercules en ce mesmes temps\\
edifia en Bourgongne une cite nomme
          en langue\\
latine au jourduy Alseta qui n'est pas loing d'Authun.\\
Oultre plus recite Diodore que les
          Galathes acom¬\\
paignes de plusieurs autres leurs vuoisins vindrent\\
en Grece occuper le pays d'Asye ou fonderent et donne¬\\
rent premierement le
          nom aux Gallathes. Ausquelz\\
sainct
            Paol a dirige et escript plusieurs epitres. Et\\
furent appellez
          ainsi que Josephe et Solinusraconte\\
et tesmoignent ceulx qui firent la
          conqueste des\\
Grecz gallatoys sunt Gallogreci. A ceste cause veult\\
inserer que d'autant que l'usaige des lettres estoit\\
entre les Galloys comme
          est dessusdit. La raison est\\
euidente a ce que dyent ceulx de Meonie qu'ilz
            auoyent\\
premierement les lectres des Francois. et que les\\
Grecz
          les eurent d'eulx et non pas de Cadmus qu'ilz\\
dient auoir este celluy qui premier les apporta\\
des Pheniciens aux Grecz
          comme ilz ont faulcement\\
de leur coustume mescongneu. Ad ce propos aussi



\newpage
sert ce que Manethon racompte que a la fin du regne\\
de Ilus Roy de Troye Galathes second de ce nom sur\\
monta les Samothes et Tanydicques et fonda plusieurs\\
villes es Gallathes par
          quoy nous est euidente pro¬\\
bation que la puissance des Galles persevera et
            con¬\\
tinua longuement en Asye. Et pour ceste cause appa¬\\
roist que
          les ars, sciences et lettres ont eu origine\\
des Francois ce que les Grecz se
          sont faulcement\\
atribues ce loz ce que iamais ne fut trouue en librai¬\\
rie ne suffisant acteur aussi que leur reprouche Joze¬\\
phus en son livre contre la grammaire Appion.
            Et\\
pour faire fin en ceste matiere Diodore dificillede Sicille dit pour\\
la verite que les Rommains appellent les Gallathes\\
ceulx que les autres nomment les Celtes qui sont les\\
Francois. Ainsi appart
          comme les francois qu'ilz\\
furent premierement nommez Samothes de leur
            premier\\
Roy muerent leur nom par plusieurs foys. Car de\\
Samothes
          apres furent appelles Celtes de leur Roy\\
Celthe. Apres furent appelles ou nommes Galathes\\
et leur regne
          augmente de Gaulle belgicque par\\
ung nomme Belgius. Et comme nous verrons\\
furent appelles
          Galles belgicques les Rommains leur\\
ont tousjours donne ce nom Galli qui sont
            Galles.\\
Et est celuy qui entre les hystoriens est plus usite\\
finablement comme nous verrons a la fin de ce\\
livre furent appelles Francois
          qui est le terme plus\\
commun de par de ca en nostre langue vulgaire. Par\\
ainsi il appart que Galathes filz de
            Hercules donna\\
ce nom Gaulles au
          pays et aux homines de France.\\
Le quel Galathes du temps de Altades XII ͤ Roy\\
 de
          Babilloyne.



\newpage






\subsection*{De Harbon XI ͤ Roy de
          Celtes}


Harbon en muant .h. en .n. est Narbon et de
            luy\\
fut dit le pays de Narbonne il estoit filz de Galathes\\
de ceste Galle de Narbonne parle Pline en son III ͤ\\
livre de l'istoire naturelle. La province dit que de\\
Narbonne est une partie des Gaulles et est au de\\
soubz de la mer
          d'Espaigne et se nommoit Galia\\
broceata diuisee de Itallie par la riviere
            nommee\\
en latin varus, il y a homme d'esperit fertilite de toute\\
multitude de richesses par quoy ne scay province\\
digne d'estre preferee au
          pays d'Ytalie sinon ceste cy.





\subsection*{De Lugdus XII ͤ Roy de
          Celtes.}


Lugdus est celluy dont la province prent
            son\\
nom comme recite Berose et ainsi que le
          nom l'accuse\\
dit maistre Jehan de Viterbe
          commentateur dudit Berose\\
c'est celuy dont est
          nomme Lugdinium qui est Lyon en\\
la Gaulle lugdunense ou lionnoise on appelle de
            cestuy\\
Lugdus les hommes Ludouici et non
          obstant g. demeure\\
Ludouici que nous disons en Francoys Loys.





\subsection*{De Belgius XIII ͤ Roy de Celtes.}


Belgius comme dit Berose est celuy dont est dicte\\
la Gaulle belgieque qui commance
          depuys le pays\\
ou passe la riviere de Cene et Marne du coste de sep\\
tentrion jusques au fleuve de Rouen et une partie\\
vers orient du coste
          d'occident finist a la mer Occeane\\
qui est la mer de Bretaigne.





\newpage
\subsection*{Du Roy Allobrox
          XIV ͤ Roy de Celtes ou Francois}


Allobrox ainsi que dit Viterboice quasi Alle¬\\
droys qui est a dire meridionnale mixtion, car\\
drox ou daros est le vent de mydy que les hebrai¬\\
ques dyent daros et
              hallo est a dire mixtion, a\\
ce consentent sainct Jherosme et
          les talmudistes\\
et de ces deux dictions vient allodarox ou allo¬\\
drox
          qui depuis a este dit Allobrox, qui est inter¬\\
prete commixtion, sont ditz les Allobroges qui ont\\
plusieurs cites tres
          anciennes comme dyent Ptholome\\
et autres
          cosmographes entre lesquelz est Vienne,\\
Geneve et Avignon meridyonale il n'est
          a dou¬\\
ter que de ce roy Allobrox.





\subsection*{Translateur}


Icy finist Berose qui a escript ceulx qui
            regnerent\\
en diverses parties de la terre apres le deluge jusques\\
a
          son temps



Maneton commance ou Berose a finy et contient\\
jusques a son temps duquel regna Francus
            dont\\
sont ditz les Francois comme l'en verra cy apres.





\subsection*{Du XV ͤ Roy des Celtes ou Francois nomme Romus}


Romus est celluy dit Manethon dont sont ditz\\
Romandi. Ptholomee dit que en sa Gaulle belgicque\\
y a les
          Romandisses sur le fleuve de Rosne au de\\
soubz de Vienne romonun et Roma qui
          depuis a\\
este nommee par interpretation Valence.







\newpage






\subsection*{Du XVI ͤ Roy de Celtes ou Francois nomme Paris}


Paris fut celluy dont est dicte la cite de
            Paris\\
qui par avant avoit nom Lutocia qui de nostre\\
temps est tres
          souueraine universite du monde en\\
la quelle fut docteur regent sainct Thomas d'Aquin.\\
Ptholomee en sa Geographie et autres plusieurs.





\subsection*{du XVII ͤ Roy de Celtes nomme Lemannus.}


Lemannus fut celuy dont sont ditz Le
            Main\\
dont est le lac que nous appellons Lusanne





\subsection*{du XVIII ͤ Roy de France nomme Olbius}


Olbius dit Maneton regna apres le Roy Mannus\\
et de luy n'est trouve chose digne de memoire.





\subsection*{du XIX ͤ Roy de Celtes ou Francois nomme Galathas\\
second de ce nom.}


Gallatas deuxieme de ce nom fut celluy
          qui surmon\\
ta les Samothes en Asye et edifia plusieurs villes\\
en
          France en l'honneur de son pere Olbius et es\\
autres regions par luy subjuguees comme en\\
Samothe, Asie et Cilicie, ainsi que
            Ptholome et\\
autres geographes nomment
          plusieurs cites\\
olbiances de ce nom Olbius.
          etc.





\end{document}
    